\subsection{Local Taxation in Ohio}

Ohio consists of 88 counties, each covering about 464 square miles (1,200 square kilometers).  Each county was historically divided into about 15 equally-sized townships, which do not cross county lines.  Citizens can petition to incorporate as a village, which has a different type of government structure than a township and the ability to levy both income and property taxes, whereas townships may only levy property taxes.  When a village exceeds 5,000 population in Ohio it is reclassified as a city.  Villages and cities may cross township and county lines, dissolve, or annex parts of contiguous townships.  Villages, cities and townships, which we call “cities” for brevity, are the most local governmental unit in Ohio. Each local government covers about 18.2 square miles (47.1 km2) on average. The Ohio Revised Code lets local governments collect a small amount of tax without a vote.  Beyond this limited amount local governments put tax levies on the ballot to ask for additional money from voters.  
The type of tax levy we study is for the renewal of infrastructure spending, which is almost entirely road maintenance.  Levies that pass typically expire, and the most common duration to collect a levy is five years, representing 89\% of the road tax levies in our sample.  If a tax levy is renewed, taxes and funding continue.  If 50\% or fewer votes approve the levy, it fails.  When a tax levy for additional funding fails, there is no increase in funding, but existing funding from other tax levies continues as normal.  When a renewal tax levy fails, funding from that tax levy stops.  99\% of the road tax levies in our sample are property taxes and 1\% are income taxes.

Most regression discontinuity studies that use voting data to look at the impact of funding changes examine new tax levies for additional funding.  \cite{cellini2010value} observes that votes for additional tax money may not be statistically independent; a vote may be proposed until it passes.  We minimize this source of endogeneity of new votes by only considering renewal votes \citep{brasington2017school}.  While a government may choose when to put a vote for additional funding on the ballot and keep proposing the new tax until it is passed, when a vote passes it has an expiration date.  So if a road tax levy for additional funding passes in 2007 to last five years, in 2012 voters will have the chance to renew or reject the tax.  The timing of the vote in 2012 is not endogenous, having been set in 2007.  If voters renew it in 2012, it will be up for renewal again in 2017. 

Road tax levies cover a broad range of activities.  Most of the renewal taxes we consider have stated purposes of “road and bridges repair”, “road repair”, “street fund”, and “street improvements”, although there are less common stated purposes like “repair and maintenance of streets and sewers system” and “resurfacing and rehabilitation of city streets.”  The construction of new roads and bridges, in contrast, would be funded with a tax levy for additional money, not a renewal tax; and it would likely be funded by a bond levy lasting 20 or 30 years.  We eliminate from our dataset stated purposes that might suggest new road construction like a 30-year 1.9-mill new tax in Moscow Village for “permanent improvements” and 0.5-mill new tax for 20 years in Shawnee Hills Village for “general construction and road and bridges repair.” Our dataset includes tax levies like a 2-mill, 5-year renewal in Adams Township (Champaign County) in 1995; a 3-mill, 5-year renewal in Lore City Village in 2016; and a 2.5-mill, 5-year renewal in Pataskala City in 2007.


\subsection{Tax Cuts \& Visual Evidence of Road Quality}

To further understand what separates the areas which regularly pass their levies from the areas which often fail to renew their levies, we identify ten cities with the largest number of successful renewals and failed renewals, which necessitate cutting funding. We share this information in Table \ref{tab:max_renewals_cuts}. 



\begin{table}[ht]
    \centering
    \caption{County Subdivisions with Most Successful and Failed Renewals}
    \label{tab:max_renewals_cuts}
    \begin{threeparttable}
    \begin{tabular}{p{4cm}p{3cm}p{3cm}cc}
        \hline
        County Subdivision & Subdivision Type & County & Renewals & Cuts \\
        \hline
        \multicolumn{5}{l}{\textbf{Panel A. Maximum number of successful renewals}} \\
        Elmore & Village & Ottawa & 20 & 0 \\
        Bainbridge & Township & Geauga & 19 & 0 \\
        Ames & Township & Athens & 17 & 0 \\
        Franklin & Township & Portage & 17 & 0 \\
        Andover & Township & Ashtabula & 16 & 0 \\
        Auburn & Township & Geauga & 16 & 0 \\
        Chester & Township & Geauga & 16 & 0 \\
        Waterloo & Township & Athens & 15 & 1 \\
        Hiram & Township & Portage & 15 & 0 \\
        Ashtabula & Township & Ashtabula & 14 & 0 \\
        \\
        \multicolumn{5}{l}{\textbf{Panel B. Maximum number of failed renewals}} \\
        Morgan & Township & Ashtabula & 7 & 9 \\
        Rush & Township & Champaign & 11 & 9 \\
        Wayne & Township & Champaign & 5 & 8 \\
        Rose & Township & Carroll & 6 & 6 \\
        Perry & Township & Morrow & 4 & 6 \\
        Washington & Township & Scioto & 10 & 6 \\
        Waynesville & Village & Warren & 2 & 6 \\
        Trumbull & Township & Ashtabula & 3 & 5 \\
        Thompson & Township & Geauga & 7 & 5 \\
        Paint & Township & Holmes & 8 & 5 \\
        \hline
    \end{tabular}
    \begin{tablenotes}
        \small
        \item This table shows the ten cities in Ohio with the largest number of renewed road maintenance levies (Panel A) and the most levies that failed to renew funding (Panel B). The table also provides subdivision type and county for each city. Townships belonging to Ashtabula county have been highlighted.  
    \end{tablenotes}
    \end{threeparttable}
\end{table}

% \begin{figure}[htbp]
%     \centering
%     \includegraphics[width=\textwidth,keepaspectratio]{images/andover_township_14grandarmy_2022.png}
% \end{figure}

\begin{figure}[htbp]
    \centering
    \begin{minipage}{0.49\textwidth}
        \centering
        \includegraphics[width=\textwidth,keepaspectratio]{images/andover_township_14grandarmy_2007.png}
        \caption*{Grand Army of the Republic Hwy: 2007} % * suppresses numbering
        \label{fig:andover_2007}
    \end{minipage}
    \hfill
    \begin{minipage}{0.49\textwidth}
        \centering
        \includegraphics[width=\textwidth,keepaspectratio]{images/andover_township_14grandarmy_2022.png}
        \caption*{Grand Army of the Republic Hwy: 2022}
        \label{fig:andover_2022}
    \end{minipage}
    \caption{Andover township - Road quality}
    \label{fig:rd_andover}
\end{figure}

\begin{figure}[htbp]
    \centering
    \begin{minipage}{0.49\textwidth}
        \centering
        \includegraphics[width=\textwidth,keepaspectratio]{images/morgan_township_2801OH-45_2009.png}
        \caption*{OH-45: 2009}
        \label{fig:morgan_oh_2009}
    \end{minipage}
    \hfill
    \begin{minipage}{0.49\textwidth}
        \centering
        \includegraphics[width=\textwidth,keepaspectratio]{images/morgan_township_2801OH-45_2018.png}
        \caption*{OH-45: 2018}
        \label{fig:morgan_oh_2018}
    \end{minipage}
    \caption{Morgan township - Road quality}
    \label{fig:rd_morgan}
\end{figure}


Table \ref{tab:max_renewals_cuts} shows that there are certain cities which always renew their road tax levies and other cities which often cut their renewal tax levies. Interestingly, we find two townships, Andover and Morgan, which belong to the same county of Ashtabula but exhibit very different behavior in terms of their voting patterns. Andover township passed all the road tax levies that were up for election whereas Morgan township failed to renew more than 56\% of its renewal levies. Since both the townships belong to the same county and have similar characteristics, we can effectively identify whether cutting road spending via road tax levies have a tangible impact on the quality of roads in the area. For this, we leverage Google Street Maps (GSM). Since photos of roads on GSM are time-stamped and the same area can have photos from different points in time, this feature allows us to observe the evolving condition of roads in these neighborhoods over time. For instance, we can observe the major streets of Andover township and compare them with the main streets of Morgan township. We checked whether there were any road tax renewals that took place between these available dates and then observed whether the road conditions improved, stayed the same or deteriorated during this period. Figure \ref{fig:rd_andover} shows Grand Army of the Republic Hwy, a specific hub in Andover, for periods 2007 and 2022. The road quality appears to have been maintained well over the years, indicating that consistent renewal of road tax levies contributes to sustained road infrastructure quality. In contrast, Figure \ref{fig:rd_morgan} illustrates the condition of OH-45 and W Water St in Morgan township for the years 2009 and 2018. The deterioration as well as continuation of poor road quality is evident, suggesting that the failure to renew road tax levies has a tangible negative impact on roads. The disparity in road conditions between Andover and Morgan townships may be suggestive of the broader economic implications of fiscal policy decisions at the local level, particularly how underinvestment in infrastructure can lead to tangible declines in public asset quality and potentially hinder economic growth and development.

% \clearpage

\subsection{Running Variable}

The running variable plays a critical role in regression discontinuity.  In our case it represents the proportion of votes against the renewal of a road tax levy.  A vote share of less than 50 means the tax will no longer be collected, and road funding from that particular tax levy will stop.  Other road tax levies may continue to be in effect, and funds from current expense tax levies may still be used for road maintenance.  There are 3,188 votes in our sample, 83\% of which renew the tax, and 17\% of which cut taxes and road maintenance.

We quantify the size of the cut in road spending a city faces in two ways.  First, not every city has a tax levy in effect specifically for roads, but every city has current expense tax funding.  Our mean 1.9 mill road tax represents 13.6\% of current expense funding, indicating that cities that fail to renew a road tax levy face a moderate reduction in funding.  We also investigate the drop in road expenditures by sampling financial audits of cities in our sample whose renewal tax levies fail, finding an average drop of \$218,808 in road spending one year later.  This \$218,808 drop represents a 34\% decrease in road spending.  
In our sample, the mean vote share against the tax levy is 0.38, which is essentially the same as the median of 0.36.  The standard deviation is 0.15, so that the typical levy that fails represents a little less than one standard deviation away from the mean. The Great Financial Crisis falls in the middle of our dataset, so readers might wonder if voting behavior was affected, but we find vote share the same to two decimal points during and outside the years 2008-2009. 


\subsection{Outcome Variables}

We study three different outcomes:  house prices, wages, and employment, although the most consistent set of treatment effects are for house prices.  

\subsubsection{Median Housing Price}

Our house price data comes from a CoreLogic® dataset of actual sales transaction prices in Ohio from 1995 through 2021 containing a little more than 7 million observations. The dependent variable, \textit{Median House Price}, reflects the median sale price of houses within a specific city and year. For example, for the houses sold in Delaware Township during the year 2002, the median sale price was \$205,041. We take precaution to only include arm’s-length transactions, and we restrict our attention to single-family residential structures for comparability.  The overall sample mean for the 10-year period from the time of vote considered in this study is \$166,082 in constant 2010 dollars with a standard deviation of \$372,135 which suggests presence of some outliers. Although our use of median sale price addresses outliers, our robustness check in Section 4.1.2 drops 1\% tails and re-estimates the treatment effects.  The mean for this winsorized sample is \$150,375 in constant 2010 dollars with a standard deviation of \$116,020. A staple feature of regression discontinuity studies is a graph of the outcome variable against the running variable. Figures \ref{fig:rd_hp_1} and \ref{fig:rd_hp_2} shows house prices from years $t+1$ to $t+8$ after the vote graphed against the percent of votes for the tax levy.

\begin{figure}[ht]
    \centering
    \begin{minipage}[b]{0.48\textwidth}
        \centering
        \includegraphics[width=\textwidth,keepaspectratio]{images/rd_plot_median_sale_amount_t_plus_1_tri_mserd_1_2_within.png}
        \caption*{Year 1 after vote}
        \label{fig:hp_year1_after}
    \end{minipage}
    \hfill
    \begin{minipage}[b]{0.48\textwidth}
        \centering
        \includegraphics[width=\textwidth,keepaspectratio]{images/rd_plot_median_sale_amount_t_plus_2_tri_mserd_1_2_within.png}
        \caption*{Year 2 after vote}
        \label{fig:hp_year2_after}
    \end{minipage}
    
    \vspace{1em}
    
    \begin{minipage}[b]{0.48\textwidth}
        \centering
        \includegraphics[width=\textwidth,keepaspectratio]{images/rd_plot_median_sale_amount_t_plus_3_tri_mserd_1_2_within.png}
        \caption*{Year 3 after vote}
        \label{fig:hp_year3_after}
    \end{minipage}
    \hfill
    \begin{minipage}[b]{0.48\textwidth}
        \centering
        \includegraphics[width=\textwidth,keepaspectratio]{images/rd_plot_median_sale_amount_t_plus_4_tri_mserd_1_2_within.png}
        \caption*{Year 4 after vote}
        \label{fig:hp_year4_after}
    \end{minipage}
    
    \caption{Median Housing Price as a function of running variable: up to year 4}
    \label{fig:rd_hp_1}
\end{figure}

\begin{figure}[ht]
    \centering
    \begin{minipage}[b]{0.48\textwidth}
        \centering
        \includegraphics[width=\textwidth,keepaspectratio]{images/rd_plot_median_sale_amount_t_plus_5_tri_mserd_1_2_within.png}
        \caption*{Year 5 after vote}
        \label{fig:hp_year5_after}
    \end{minipage}
    \hfill
    \begin{minipage}[b]{0.48\textwidth}
        \centering
        \includegraphics[width=\textwidth,keepaspectratio]{images/rd_plot_median_sale_amount_t_plus_6_tri_mserd_1_2_within.png}
        \caption*{Year 6 after vote}
        \label{fig:hp_year6_after}
    \end{minipage}
    
    \vspace{1em}
    
    \begin{minipage}[b]{0.48\textwidth}
        \centering
        \includegraphics[width=\textwidth,keepaspectratio]{images/rd_plot_median_sale_amount_t_plus_7_tri_mserd_1_2_within.png}
        \caption*{Year 7 after vote}
        \label{fig:hp_year7_after}
    \end{minipage}
    \hfill
    \begin{minipage}[b]{0.48\textwidth}
        \centering
        \includegraphics[width=\textwidth,keepaspectratio]{images/rd_plot_median_sale_amount_t_plus_8_tri_mserd_1_2_within.png}
        \caption*{Year 8 after vote}
        \label{fig:hp_year8_after}
    \end{minipage}
    
    \caption{Median Housing Price as a function of running variable: year 5 to year 8}
    \label{fig:rd_hp_2}
\end{figure}

Scatterplots like Figure \ref{fig:rd_hp_1} and \ref{fig:rd_hp_2}, although expected in regression discontinuity studies, should be read with caution, because the dots in the graphs do not represent actual house prices or vote shares.  A graph of actual data would illustrate a mess of dots for which no human could see a discernible pattern.  Graphing involves averaging vote shares into bins representing a narrow range of vote shares and summarizing the wide range of house prices within each bin with a single mean value.  For this reason, graphs do not necessarily track treatment effects that come from formal regression analysis using actual individual data points.  Nevertheless, the graphs in Figure \ref{fig:rd_hp_1} and \ref{fig:rd_hp_2} representing 8 years after the vote may show a discount in house prices from cutting taxes and road maintenance. Corresponding estimates and graphs for treatment effects in the full set of years $t+1$ through $t+10$ relative to the vote can be found in Figure \ref{fig:tes_gs_app} and Table \ref{tab:median_sale_amount_full} in the Appendix.

\subsubsection{Employment and Wages}

We obtain special permission from the Ohio Department of Jobs and Family Services (ODJFS) to access establishment-level data on wages and employment.  It shows total wages and the average number of employees for each quarter.  We aggregate up from quarterly to yearly values.  Each establishment has an address, which we geocode to the proper local government jurisdiction (city, village or township), and we aggregate total wages and average employment up to the local government level to match the unit of geography of the voting data. The outcome variables considered were “Average employment” and “Wages per capita”, where average employment is the average number of employees on the payroll in a particular city during a specific year, and wages per capita is aggregate total wages paid to the employees on the payroll in particular city during a year divided by average employment. Graphs of these outcome variables are available upon request.  Within the typical effective bandwidth of 14 percentage points, mean city-level employment is 1,575 for the fail levy group with a standard deviation of 3,319; and it is 1,342 with a standard deviation of 3,514 for the pass levy group the year before the vote.  

The average wage outcome we study also has similar means within the effective bandwidth the year before the vote.  For the high-poverty sample, the sample with the only significant treatment effects, mean wages in 2010 dollars are \$29,153 for the fail-levy sample with a standard deviation of 8,515; and they are \$29,968 for the pass-levy sample with a standard deviation of 7,524.

\subsection{Covariates}

Covariates can be useful in regression discontinuity studies, although they are not necessary for the identification of treatment effects.  One use of covariates is to increase the precision of treatment effect estimates.  The other is to see if cities that barely pass and fail tax levies are similar to each other like the theory of regression discontinuity says they should be. Table \ref{tab:variable_means_sd} shows covariate means for both the global sample of all votes in the data set as well as the local sample within a representative effective bandwidth of the 0.50 cutoff. The effective bandwidth displayed in Table \ref{tab:variable_means_sd} is the mean bandwidth for all the housing outcome regressions. The first columns for the global sample show similar values of characteristics between cities that renew and cut road taxes and spending, but it is the two rightmost columns that are critical for the credibility of the regression discontinuity design.   

Table \ref{tab:variable_means_sd} demonstrates the covariate balance between cities that renew their tax levies and those that do not, indicating similar demographic and economic profiles across both groups. The data, captured at the time of the vote, shows minimal differences within the effective bandwidth: the mean population differs by only 224 from a base of 5,220, and median family income varies by \$190 from a base of \$60,269, measured in 2010 U.S. dollars. Other variables, including poverty rates, single-parent household percentages, educational attainment, age distribution, and racial composition, show differences of less than one percentage point, reinforcing the comparability of the two groups.


\begin{longtable}{p{4cm}cccccc}
    \caption{Variable Means \& Standard Deviation by Tax Levy Renewal Status} \label{tab:variable_means_sd} \\
    \hline
    Variable & \multicolumn{3}{c}{Global} & \multicolumn{2}{c}{Effective} \\
    \cmidrule(lr){2-4} \cmidrule(lr){5-6}
    & Full Sample & Renewed & Cut & Renewed (Control) & Cut (Treatment) \\
    \hline
    \endfirsthead

    \multicolumn{6}{c}{{\bfseries \tablename\ \thetable{} -- continued from previous page}} \\
    \hline
    Variable & \multicolumn{3}{c}{Global} & \multicolumn{2}{c}{Effective} \\
    \cmidrule(lr){2-4} \cmidrule(lr){5-6}
    & Full Sample & Renewed & Cut & Renewed (Control) & Cut (Treatment) \\
    \hline
    \endhead

    \hline \multicolumn{6}{r}{{Continued on next page}} \\
    \endfoot

    \hline
    \endlastfoot

    \multicolumn{6}{l}{\textbf{Panel A. Covariates}} \\
    Population & 5,074 & 5,134 & 4,769 & 5,220 & 4,996 \\
               & (7,942) & (8,030) & (7,478) & (8,205) & (7,420) \\
    Poverty Rate & 0.11 & 0.11 & 0.11 & 0.11 & 0.10 \\
                 & (0.08) & (0.08) & (0.07) & (0.08) & (0.07) \\
    \% with Kids & 0.39 & 0.39 & 0.40 & 0.39 & 0.39 \\
                 & (0.08) & (0.08) & (0.08) & (0.08) & (0.07) \\
    \% Households with Children under 18 & 0.09 & 0.09 & 0.09 & 0.09 & 0.08 \\
                                         & (0.06) & (0.06) & (0.05) & (0.06) & (0.05) \\
    \% Less than High School Education & 0.16 & 0.15 & 0.18 & 0.17 & 0.18 \\
                                       & (0.11) & (0.11) & (0.10) & (0.11) & (0.09) \\
    \% Some College Education & 0.25 & 0.25 & 0.24 & 0.25 & 0.24 \\
                              & (0.06) & (0.06) & (0.06) & (0.06) & (0.07) \\
    \% Unemployment Rate & 0.05 & 0.05 & 0.05 & 0.06 & 0.05 \\
                         & (0.04) & (0.04) & (0.03) & (0.04) & (0.03) \\
    \% Renters & 0.20 & 0.20 & 0.20 & 0.20 & 0.19 \\
               & (0.11) & (0.11) & (0.10) & (0.11) & (0.09) \\
    \% White & 0.96 & 0.96 & 0.97 & 0.97 & 0.96 \\
             & (0.07) & (0.08) & (0.07) & (0.08) & (0.08) \\
    \% Black & 0.02 & 0.02 & 0.02 & 0.02 & 0.02 \\
             & (0.07) & (0.07) & (0.06) & (0.07) & (0.07) \\
    \% Married & 0.59 & 0.59 & 0.60 & 0.60 & 0.61 \\
               & (0.09) & (0.09) & (0.08) & (0.09) & (0.08) \\
    \% Separated & 0.01 & 0.01 & 0.01 & 0.01 & 0.01 \\
                 & (0.01) & (0.01) & (0.01) & (0.01) & (0.01) \\
    Income Heterogeneity Index & 0.10 & 0.10 & 0.09 & 0.09 & 0.09 \\
                               & (0.08) & (0.08) & (0.07) & (0.07) & (0.06) \\
    Median Family Income & 61,014 & 61,445 & 58,837 & 60,269 & 60,079 \\
                         & (17,694) & (18,283) & (14,177) & (15,687) & (14,143) \\
    \% Under 5 Years Old & 0.06 & 0.06 & 0.06 & 0.06 & 0.06 \\
                         & (0.02) & (0.02) & (0.02) & (0.02) & (0.02) \\
    \% Aged 5 to 17 & 0.20 & 0.20 & 0.21 & 0.20 & 0.20 \\
                    & (0.05) & (0.05) & (0.04) & (0.05) & (0.04) \\
    \% Aged 18 to 64 & 0.60 & 0.60 & 0.60 & 0.60 & 0.60 \\
                     & (0.05) & (0.05) & (0.05) & (0.05) & (0.04) \\
    \% Racial Minority & 0.04 & 0.04 & 0.03 & 0.03 & 0.04 \\
                       & (0.07) & (0.08) & (0.07) & (0.08) & (0.08) \\
    Number of Observations & 3,188 & 2,661 & 527 & 694 & 273 \\
    \\
    \multicolumn{6}{l}{\textbf{Panel B. Outcome Variables}} \\
    \textit{a) Median House Price} & & & & & \\
    t - 3 & 106,067 & 107,502 & 102,766 & 106,771 & 104,201 \\
          & (52,988) & (60,197) & (50,794) & (53,461) & (51,794) \\
    t - 2 & 105,528 & 107,211 & 101,456 & 106,207 & 103,713 \\
          & (53,242) & (60,893) & (48,438) & (54,162) & (50,786) \\
    t - 1 & 107,706 & 108,234 & 104,575 & 108,401 & 105,843 \\
          & (53,544) & (62,437) & (51,343) & (54,665) & (50,491) \\
    \textit{b) Employment} & & & & & \\
    t - 3 & 1,504 & 1,752 & 1,728 & 1,446 & 1,695 \\
          & (3,848) & (4,006) & (3,642) & (3,887) & (3,739) \\
    t - 2 & 1,429 & 1,734 & 1,680 & 1,388 & 1,562 \\
          & (3,693) & (3,978) & (3,612) & (3,769) & (3,462) \\
    t - 1 & 1,387 & 1,724 & 1,668 & 1,341 & 1,538 \\
          & (3,628) & (3,993) & (3,598) & (3,683) & (3,460) \\
    \textit{c) Wages per Capita} & & & & & \\
    t - 3 & 31,764 & 31,892 & 32,531 & 32,367 & 32,401 \\
          & (8,868) & (8,428) & (10,090) & (9,657) & (8,898) \\
    t - 2 & 31,714 & 32,083 & 32,048 & 31,571 & 32,166 \\
          & (8,630) & (8,392) & (9,224) & (8,723) & (8,372) \\
    t - 1 & 31,764 & 32,399 & 32,399 & 31,579 & 32,367 \\
          & (8,868) & (9,716) & (9,716) & (8,623) & (9,657) \\
\end{longtable}

% \begin{tablenotes}
%     \small
%     \item Notes: Panel A shows covariates used in estimating the treatment effect of failing to renew a road tax levy. Covariate means at the time of the vote are shown along with standard deviations in parentheses. The unit of observation is city-year. 'Income' is median family income in constant 2010 U.S. dollars. 'Global' refers to the full sample of city-year votes; 'Effective' refers to city-year votes within the mean estimated optimal bandwidth of 10 percentage points around the cutoff for years 1995 through 2021. The treatment group reflects cutting road funding by failing to renew the road tax levy. The control group reflects maintaining road funding by renewing the road tax levy. Panel B shows means of outcome variables for placebo years t - 3, t - 2, and t - 1.
% \end{tablenotes}


Table \ref{tab:variable_means_sd} shows that the treatment and control groups are well-balanced with respect to the covariates considered, suggesting that the groups are comparable and any observed differences in outcomes can more confidently be attributed to the treatment effect rather than to pre-existing differences.

\subsection{Testing how well the data meets the assumptions of RDD}

The assumptions of regression discontinuity are minimal compared to other identification strategies.  One is that no agent, while having some influence, may precisely assign observations to a specific value of the running variable.  In our context, it means that higher levels of government, foreign governments, or the firm that programs the voting machines may not dictate vote share for city road tax levies.  The standard way to test this assumption is to perform a density test like that of \cite{cattaneo2020simple}, which is based on the idea that a manipulating agent might cause a clustering of votes just to one side of the cutoff, with a pronounced drop-off on the other side of the cutoff.  The $p$-value of this density test is 0.98. A histogram of vote share is shown in Figure \ref{fig:running_var_hist} that graphically illustrates the lack of abrupt change in density.  

\begin{figure}[ht]
    \centering
    \includegraphics[width=\textwidth,keepaspectratio]{images/votes_pct_against_histogram.png}
    \caption{Histogram of Running Variable}
    \label{fig:running_var_hist}
\end{figure}

Although Table \ref{tab:variable_means_sd} shows covariate balance between sets of cities that pass and fail to renew road tax levies, covariate values could still jump from one side of the cutoff to another. A drop in education levels, for example, could cause a drop in house prices that might coincide with a change in treatment, so that what might look like a treatment effect from cutting taxes and spending might in fact be caused by lower education levels.  Graphs of covariate smoothness around the cutoff are found in the Appendix \ref{sec:appxb}. A formal way to assess covariate smoothness is to use each covariate as an outcome variable in a regression of the running variable and a treatment effect dummy. When we do so, the $p$-value of the treatment effect dummy varies from 0.13 to 0.98, indicating no statistically significant jump in covariate values. 
