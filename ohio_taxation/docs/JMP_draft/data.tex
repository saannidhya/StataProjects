\subsection{How are roads funded in Ohio?}

Roads in Ohio are funded through a combination of federal, state, and local sources. A significant portion of road funding provided by federal and state governments comes from gas taxes, which are currently set at $\$0.18$ per gallon for federal tax and $\$0.38$ per gallon for Ohio state tax. Additional sources for these 2 levels of government include vehicle registration fees, license plate fees, tolls, and driver's license fees. Funding for local governments largely comes from property taxes. Below, we provide an overview of road funding in Ohio.

{\bf Federal Funding.} U.S Department of Transportation (USDOT) provides funding for road infrastructure through the Federal Highway Administration (FHWA). The FHWA provides funding for the construction, maintenance, and operation of highways, bridges, and tunnels. The federal government provides funding for road infrastructure through the Highway Trust Fund (HTF), which is funded by the federal gas tax. The HTF is divided into two accounts: the Highway Account and the Mass Transit Account. The Highway Account is used to fund highway construction and maintenance, while the Mass Transit Account is used to fund public transportation projects. The federal government also provides funding for road infrastructure through the Surface Transportation Block Grant Program (STBG) or Bipartisan Infrastructure Law \citep{USDOT_BIL_Ohio}, which provides funding for road projects that are not eligible for funding through the HTF. Nevertheless, federal government funding for road infrastructure is generally limited, as in 2022, about 4/5$^\text{th}$ of the funding came from state and local governments \citep{pgpf_htf_explained}.

{\bf State Funding.}
In Ohio, gas taxes, licensing fees and user fees account for $69\%$ of the state's road funding \citep{boesen2021}. The Ohio Department of Transportation (ODOT) is responsible for the allocation of state funds, which are used for both maintenance and new construction. However, about 70\% of funds go to new highway construction, 2\% is given to local governments as grants and only 4\% of state funds is directed towards local road maintenance \citep{odot2023}. The remaining funds are part of payroll \& operating expenses, and other miscellaneous expenses. Hence, most of the road maintenance funding for local, neighborhood-specific roads in Ohio comes from local governments.


{\bf Local Funding.} Local governments in Ohio fund roads mainly through a portion of property taxes, even though there is variation in such funding across the local governments. These municipalities have the authority to levy property taxes specifically for road maintenance, providing a crucial source of funding for the upkeep of local infrastructure. As per our correspondance with the Beavercreek Township, 61\% of their road funds come from property taxes, and only 8\% from gas taxes. Moreover, 77\% of funding for roads in Beavercreek township is provided by the local government and 11\% of overall local government budget is allocated to roadway maintenance \citep{schroeder2025}. Hence, local roads are primarily funded by local governments in Ohio and road maintenance funding is a significant part of the local government's budget.


\subsection{Local Taxation in Ohio}

{\bf Background.} Ohio consists of 88 counties, each covering about 464 miles$^2$ (1,200 kilometers$^2$). Each county was historically divided into about 15 equally-sized townships, which do not cross county lines. Citizens can petition to incorporate as a village, which has a different type of government structure than a township and the ability to levy both income and property taxes, whereas townships may only levy property taxes. When a village exceeds 5,000 population in Ohio, it is reclassified as a city. Villages and cities may cross township and county lines, dissolve, or annex parts of contiguous townships. Villages, cities and townships, which we call “cities” for brevity, are the most local governmental unit in Ohio. Each local government covers about 18.2 miles$^2$ (47.1 km$^2$) on average. The Ohio Revised Code lets local governments collect a small amount of tax without a vote. Beyond this limited amount local governments put tax levies on the ballot to ask for additional money from voters. Ohio has 3,939 local governments in total as of 2022 \citep{stlouisfed2024local}.  

The type of tax levy we study is for the renewal of road tax levy, which covers a broad range of activities. Most of the renewal taxes we consider have stated purposes of “road and bridges repair”, “road repair”, “street fund”, and “street improvements”, although there are less common stated purposes like “repair and maintenance of streets and sewers system” and “resurfacing and rehabilitation of city streets.”  The construction of new roads and bridges, in contrast, would be funded with a tax levy for additional money, not a renewal tax; and it would likely be funded by a bond levy lasting 20 or 30 years.  We eliminate from our dataset stated purposes that might suggest new road construction like a 30-year 1.9-mill new tax in Moscow Village for “permanent improvements” and 0.5-mill new tax for 20 years in Shawnee Hills Village for “general construction and road and bridges repair.” Our dataset includes tax levies like a 2-mill, 5-year renewal in Adams Township (Champaign County) in 1995; a 3-mill, 5-year renewal in Lore City Village in 2016; and a 2.5-mill, 5-year renewal in Pataskala City in 2007.

\noindent Levies that had originally passed typically expire, and the most common duration to collect a levy is five years, representing 89\% of the road tax levies in our sample.  If a tax levy is renewed, taxes and funding continue.  If 50\% or fewer votes approve the levy, it fails.  When a tax levy for additional funding fails, there is no increase in funding, but existing funding from other tax levies continues as normal.  When a renewal tax levy fails, funding from that tax levy stops.  99\% of the road tax levies in our sample are property taxes and 1\% are income taxes.

{\bf Renewal levies.} Most regression discontinuity studies that use voting data to look at the impact of funding changes examine new tax levies for additional funding.  \cite{cellini2010value} observes that votes for additional tax money may not be statistically independent; a vote may be proposed until it passes.  We minimize this source of endogeneity of new votes by only considering renewal votes \citep{brasington2017school}.  While a government may choose when to put a vote for additional funding on the ballot and keep proposing the new tax until it is passed, when a vote passes it has an expiration date.  So if a road tax levy for additional funding passes in 2007 to last five years, in 2012 voters will have the chance to renew or reject the tax.  The timing of the vote in 2012 is not endogenous, having been set in 2007.  If voters renew it in 2012, it will be up for renewal again in 2017. 

{\bf Consequences of failing to renew a levy.}


\subsection{Tax Cuts \& Evidence of Road Quality}

% \subsubsection{Visual Evidence of Road Quality}

\subsubsection{A Tale of Two Cities}

To further understand what differentiates the areas which regularly pass their levies from the areas which often fail to renew their levies, we present two townships in the same county that exhibit very different behavior with respect to successful renewal of their road tax levies. Table \ref{tab:renewals_cuts} provides a comparative analysis of two townships, Andover and Morgan, within Ashtabula County. Despite being in the same county, these townships show contrasting patterns in the renewal of road tax levies. Andover township has consistently passed all its road tax levies, while Morgan township has failed to renew more than half of its levies. This stark difference in voting behavior allows us to examine the impact of road tax levy renewals on road quality within a controlled environment, as both townships share similar geographic,  demographic and economic characteristics.

\begin{table}[ht]
    \centering
    \caption{Andover \& Morgan Townships - Referendum Results}
    \label{tab:renewals_cuts}
    \begin{threeparttable}
    \begin{tabular}{p{4cm}p{3cm}p{3cm}cc}
        \hline
        \multicolumn{5}{l}{\textbf{Panel A. Levy cut Rates}} \\
        County Subdivision  & County & Renewals & Cuts & \% Cut  \\
        \hline
        % Elmore & Village & Ottawa & 20 & 0 \\
        % Bainbridge & Township & Geauga & 19 & 0 \\
        % Ames & Township & Athens & 17 & 0 \\
        % Franklin & Township & Portage & 17 & 0 \\
        Andover & Ashtabula & 16 & 0 & 0\% \\
        Morgan & Ashtabula & 7 & 9 & 56\% \\
        % Auburn & Township & Geauga & 16 & 0 \\
        % Chester & Township & Geauga & 16 & 0 \\
        % Waterloo & Township & Athens & 15 & 1 \\
        % Hiram & Township & Portage & 15 & 0 \\
        % Ashtabula & Township & Ashtabula & 14 & 0 \\
        \\
        \hline
        \multicolumn{5}{l}{\textbf{Panel B. Township Characteristics}} \\
        Covariates & Andover & Morgan & Renewals & Cuts \\
        \hline
        Morgan & Township & Ashtabula & 7 & 9 \\
        Rush & Township & Champaign & 11 & 9 \\
        Wayne & Township & Champaign & 5 & 8 \\
        Rose & Township & Carroll & 6 & 6 \\
        Perry & Township & Morrow & 4 & 6 \\
        Washington & Township & Scioto & 10 & 6 \\
        Waynesville & Village & Warren & 2 & 6 \\
        Trumbull & Township & Ashtabula & 3 & 5 \\
        Thompson & Township & Geauga & 7 & 5 \\
        Paint & Township & Holmes & 8 & 5 \\
        \hline
    \end{tabular}
    \begin{tablenotes}
        \small
        \item This table shows the ten cities in Ohio with the largest number of renewed road maintenance levies (Panel A) and the most levies that failed to renew funding (Panel B). The table also provides subdivision type and county for each city. Townships belonging to Ashtabula county have been highlighted.  
    \end{tablenotes}
    \end{threeparttable}
\end{table}

% \begin{figure}[htbp]
%     \centering
%     \includegraphics[width=\textwidth,keepaspectratio]{images/andover_township_14grandarmy_2022.png}
% \end{figure}

\begin{figure}[htbp]
    \centering
    \begin{minipage}{0.49\textwidth}
        \centering
        \includegraphics[width=\textwidth,keepaspectratio]{images/andover_township_14grandarmy_2007.png}
        \caption*{Grand Army of the Republic Hwy: 2007} % * suppresses numbering
        \label{fig:andover_2007}
    \end{minipage}
    \hfill
    \begin{minipage}{0.49\textwidth}
        \centering
        \includegraphics[width=\textwidth,keepaspectratio]{images/andover_township_14grandarmy_2022.png}
        \caption*{Grand Army of the Republic Hwy: 2022}
        \label{fig:andover_2022}
    \end{minipage}
    \caption{Andover township - Road quality}
    \label{fig:rd_andover}
\end{figure}

\begin{figure}[htbp]
    \centering
    \begin{minipage}{0.49\textwidth}
        \centering
        \includegraphics[width=\textwidth,keepaspectratio]{images/morgan_township_2801OH-45_2009.png}
        \caption*{OH-45: 2009}
        \label{fig:morgan_oh_2009}
    \end{minipage}
    \hfill
    \begin{minipage}{0.49\textwidth}
        \centering
        \includegraphics[width=\textwidth,keepaspectratio]{images/morgan_township_2801OH-45_2018.png}
        \caption*{OH-45: 2018}
        \label{fig:morgan_oh_2018}
    \end{minipage}
    \caption{Morgan township - Road quality}
    \label{fig:rd_morgan}
\end{figure}


Table \ref{tab:renewals_cuts} shows that there are certain cities which always renew their road tax levies and other cities which often cut their renewal tax levies. Interestingly, we find two townships, Andover and Morgan, which belong to the same county of Ashtabula but exhibit very different behavior in terms of their voting patterns. Andover township passed all the road tax levies that were up for election whereas Morgan township failed to renew more than 56\% of its renewal levies. Since both the townships belong to the same county and have similar characteristics, we can effectively identify whether cutting road spending via road tax levies have a tangible impact on the quality of roads in the area. For this, we leverage Google Street Maps (GSM). Since photos of roads on GSM are time-stamped and the same area can have photos from different points in time, this feature allows us to observe the evolving condition of roads in these neighborhoods over time. For instance, we can observe the major streets of Andover township and compare them with the main streets of Morgan township. We checked whether there were any road tax renewals that took place between these available dates and then observed whether the road conditions improved, stayed the same or deteriorated during this period. Figure \ref{fig:rd_andover} shows Grand Army of the Republic Hwy, a specific hub in Andover, for periods 2007 and 2022. The road quality appears to have been maintained well over the years, indicating that consistent renewal of road tax levies contributes to sustained road infrastructure quality. In contrast, Figure \ref{fig:rd_morgan} illustrates the condition of OH-45 and W Water St in Morgan township for the years 2009 and 2018. The deterioration as well as continuation of poor road quality is evident, suggesting that the failure to renew road tax levies has a tangible negative impact on roads. The disparity in road conditions between Andover and Morgan townships may be suggestive of the broader economic implications of fiscal policy decisions at the local level, particularly how underinvestment in infrastructure can lead to tangible declines in public asset quality and potentially hinder economic growth and development.

\subsubsection{Assessing Road Quality using Satellite Imagery} \label{sec:road_quality}

To understand the effect of cutting local road tax on road quality, we fine-tune a Vision Transformer (ViT) model, gpt-4o by OpenAI, on satellite imagery data from \cite{brewer2021}. Following the seminal work on text-based Transformers in Natural Language Processing (NLP) by \cite{vaswani2017attention}, ViTs were introduced by Google Brain's team and are part of a recent class of deep learning models that have shown to outperform Convolutional Neural Networks (CNNs) on image classification tasks \citep{dosovitskiy2020image}. Below, we outline the steps we take to assess road quality using the ViT model.


% We use the ViT model to classify road images into three categories: good, average, and poor. We then use the classification results to assess the quality of roads in different neighborhoods in Ohio. We collect road images for areas that were within the average effective RD bandwidth provided in Table \ref{tab:median_sale_amount} and ensure that we have pre and post referendum images for both, the treatment and control groups. This allows us to construct a novel database of road images and assess the impact of cutting local road tax on road quality.

{\bf Satellite Images for fine-tuning.} In order to enable ViTs to accurately predict road quality, we needed large-scale road-image data to fine-tune OpenAI's gpt-4o, a Large Language Model (LLM) with multi-modal ability that has shown to outperform various other models and benchmarks \citep{achiam2023gpt}. We use the road-image dataset by \cite{brewer2021} which consists of 53,677 labeled images of road infrastructure for U.S. from 1,000 cities across the United States. The dataset includes images of roads in different conditions, and a classification representing quality of the road: 0 (poor), 1 (decent) and 2 (high). 

{\bf Ohio Satellite Images.} We use satellite images from Google Earth Pro for roads in different neighborhoods in Ohio. We collect road images for areas that were within the average effective RD bandwidth provided in Table \ref{tab:median_sale_amount} and ensure that we have pre and post referendum images for both, the treatment and control groups. This allows us to construct a novel database of road images, which we use to classify road quality for areas that were part of our quasi-experiment. 


{\bf Fine-tuning process.} Fine-tuning a ViT involves taking a pretrained model and adapting it for specific image-classification task. We fine-tune gpt-4o, a pretrained model with over 200 billion parameters, on the road-image dataset by \cite{brewer2021} to classify road images into three categories: 0, 1 and 2. Next, we divide the data into traning and validation datasets. Using OpenAI's API, we set up a training seed, convert the images into their corresponding base64 encoding and fine-tune the model on the road-image dataset for 10 epochs with a batch size of 9 and a learning rate multiplier of 2. For further details on fine-tuning and results, see Appendix \ref{sec:appxc1}. 

% We use the classification results to assess the quality of roads in different neighborhoods in Ohio. We collect road images for areas that were within the average effective RD bandwidth provided in Table \ref{tab:median_sale_amount} and ensure that we have pre and post referendum images for both, the treatment and control groups. This allows us to construct a novel database of road images and assess the impact of cutting local road tax on road quality.

% {\bf Predicting Road Quality.} 




% \bf{Analyzing satellite images.} Following \cite{brewer2021}

% \bf{Street Images.} We utilize a 

% \clearpage

\subsection{Road Maintenance Funding}



\subsection{Running Variable}

The running variable plays a critical role in regression discontinuity.  In our case it represents the proportion of votes against the renewal of a road tax levy.  A vote share of less than 50 means the tax will no longer be collected, and road funding from that particular tax levy will stop.  Other road tax levies may continue to be in effect, and funds from current expense tax levies may still be used for road maintenance.  There are 3,188 votes in our sample, 83\% of which renew the tax, and 17\% of which cut taxes and road maintenance.

We quantify the size of the cut in road spending a city faces in two ways.  First, not every city has a tax levy in effect specifically for roads, but every city has current expense tax funding.  Our mean 1.9 mill road tax represents 13.6\% of current expense funding, indicating that cities that fail to renew a road tax levy face a moderate reduction in funding.  We also investigate the drop in road expenditures by sampling financial audits of cities in our sample whose renewal tax levies fail, finding an average drop of \$218,808 in road spending one year later.  This \$218,808 drop represents a 34\% decrease in road spending.  
In our sample, the mean vote share against the tax levy is 0.38, which is essentially the same as the median of 0.36.  The standard deviation is 0.15, so that the typical levy that fails represents a little less than one standard deviation away from the mean. The Great Financial Crisis falls in the middle of our dataset, so readers might wonder if voting behavior was affected, but we find vote share the same to two decimal points during and outside the years 2008-2009. 

{\bf Density test.} Classic regression discontinuity design (RDD) assumption states that agents cannot precisely manipulate the running variable to fall just above or below the cutoff. In our context, it means that the election results are not determined prior to when the ballot takes place. In other words, no individuals, organizations, higher levels of government, foreign governments, or the firm that programs the voting machines are dictating the precise vote share for the renewal road tax levy referendums raised by a city. The standard way to test this assumption is to perform a density test like that of \cite{cattaneo2020simple}, which is based on the idea that manipulation of elections might cause a clustering of votes just to one side of the cutoff, with a pronounced drop-off on the other side of the cutoff. The $p$-value of this density test is 0.98. A histogram of vote share is shown in Figure \ref{fig:running_var_hist} that graphically illustrates the lack of abrupt change in density.

\begin{figure}[ht]
    \centering
    \includegraphics[width=\textwidth,keepaspectratio]{images/votes_pct_against_histogram.png}
    \caption{Histogram of Running Variable}
    \label{fig:running_var_hist}
\end{figure}

Although Table \ref{tab:variable_means_sd} shows covariate balance between sets of cities that pass and fail to renew road tax levies, covariate values could still jump from one side of the cutoff to another. A drop in education levels, for example, could cause a drop in house prices that might coincide with a change in treatment, so that what might look like a treatment effect from cutting taxes and spending might in fact be caused by lower education levels.  Graphs of covariate smoothness around the cutoff are found in the Appendix \ref{sec:appxb}. A formal way to assess covariate smoothness is to use each covariate as an outcome variable in a regression of the running variable and a treatment effect dummy. When we do so, the $p$-value of the treatment effect dummy varies from 0.13 to 0.98, indicating no statistically significant jump in covariate values. 



\subsection{Outcome Variables}

{\bf Median Housing Prices.} Our house price data comes from a CoreLogic® dataset of actual sales transaction prices in Ohio from 1995 through 2021 containing over 7 million observations. The dependent variable, \textit{Median House Price}, reflects the median sale price of houses within a specific city and year. For example, for the houses sold in Delaware Township during the year 2002, the median sale price was \$205,041. We take precaution to only include arm’s-length transactions, and we restrict our attention to single-family residential structures for comparability.  The overall sample mean for the 10-year period from the time of vote considered in this study is \$166,082 in constant 2010 dollars with a standard deviation of \$372,135 which suggests presence of some outliers. Although our use of median sale price addresses outliers, one of our robustness checks drops 1\% tails and re-estimates the treatment effects.  The mean for this winsorized sample is \$150,375 in constant 2010 dollars with a standard deviation of \$116,020. 

\begin{figure}[ht]
    \centering
    \includegraphics[width=0.8\textwidth,keepaspectratio]{images/rd_plot_year_5_after_vote.png}
    \caption{Median Sale Price of Homes: 5 years after vote}
    \label{fig:hp_year5_after}
\end{figure}

Figure \ref{fig:hp_year5_after} shows house prices from 5 years after the vote graphed against the percent of votes for the tax levy. The points represent the mean house price for the 10 representative bins of vote shares for the average effective bandwidth around the cutoff of 50\%. The graph shows a clear discontinuity in house prices at the cutoff, which is the basis for our identification strategy. We observe a sharp decline after the cutoff in house prices, indicating that the failure to renew a road tax levy has a negative impact on house prices.


\subsection{Covariates}

Covariates can be useful in regression discontinuity studies, although they are not necessary for the identification of treatment effects.  One use of covariates is to increase the precision of treatment effect estimates.  The other is to see if cities that barely pass and fail tax levies are similar to each other like the theory of regression discontinuity says they should be. Table \ref{tab:variable_means_sd} shows covariate means for both the global sample of all votes in the data set as well as the local sample within a representative effective bandwidth of the 0.50 cutoff. The effective bandwidth displayed in Table \ref{tab:variable_means_sd} is the mean bandwidth for all the housing outcome regressions. The first columns for the global sample show similar values of characteristics between cities that renew and cut road taxes and spending, but it is the two rightmost columns that are critical for the credibility of the regression discontinuity design.   

Table \ref{tab:variable_means_sd} demonstrates the covariate balance between cities that renew their tax levies and those that do not, indicating similar demographic and economic profiles across both groups. The data, captured at the time of the vote, shows minimal differences within the effective bandwidth: the mean population differs by only 224 from a base of 5,220, and median family income varies by \$190 from a base of \$60,269, measured in 2010 U.S. dollars. Other variables, including poverty rates, single-parent household percentages, educational attainment, age distribution, and racial composition, show differences of less than one percentage point, reinforcing the comparability of the two groups.


\begin{longtable}{p{4cm}cccccc}
    \caption{Variable Means \& Standard Deviation by Tax Levy Renewal Status} \label{tab:variable_means_sd} \\
    \hline
    Variable & \multicolumn{3}{c}{Global} & \multicolumn{2}{c}{Effective} \\
    \cmidrule(lr){2-4} \cmidrule(lr){5-6}
    & Full Sample & Renewed & Cut & Renewed (Control) & Cut (Treatment) \\
    \hline
    \endfirsthead

    \multicolumn{6}{c}{{\bfseries \tablename\ \thetable{} -- continued from previous page}} \\
    \hline
    Variable & \multicolumn{3}{c}{Global} & \multicolumn{2}{c}{Effective} \\
    \cmidrule(lr){2-4} \cmidrule(lr){5-6}
    & Full Sample & Renewed & Cut & Renewed (Control) & Cut (Treatment) \\
    \hline
    \endhead

    \hline \multicolumn{6}{r}{{Continued on next page}} \\
    \endfoot

    \hline
    \endlastfoot

    \multicolumn{6}{l}{\textbf{Panel A. Covariates}} \\
    Population & 5,074 & 5,134 & 4,769 & 5,220 & 4,996 \\
               & (7,942) & (8,030) & (7,478) & (8,205) & (7,420) \\
    Poverty Rate & 0.11 & 0.11 & 0.11 & 0.11 & 0.10 \\
                 & (0.08) & (0.08) & (0.07) & (0.08) & (0.07) \\
    \% with Kids & 0.39 & 0.39 & 0.40 & 0.39 & 0.39 \\
                 & (0.08) & (0.08) & (0.08) & (0.08) & (0.07) \\
    \% Households with Children under 18 & 0.09 & 0.09 & 0.09 & 0.09 & 0.08 \\
                                         & (0.06) & (0.06) & (0.05) & (0.06) & (0.05) \\
    \% Less than High School Education & 0.16 & 0.15 & 0.18 & 0.17 & 0.18 \\
                                       & (0.11) & (0.11) & (0.10) & (0.11) & (0.09) \\
    \% Some College Education & 0.25 & 0.25 & 0.24 & 0.25 & 0.24 \\
                              & (0.06) & (0.06) & (0.06) & (0.06) & (0.07) \\
    \% Unemployment Rate & 0.05 & 0.05 & 0.05 & 0.06 & 0.05 \\
                         & (0.04) & (0.04) & (0.03) & (0.04) & (0.03) \\
    \% Renters & 0.20 & 0.20 & 0.20 & 0.20 & 0.19 \\
               & (0.11) & (0.11) & (0.10) & (0.11) & (0.09) \\
    \% White & 0.96 & 0.96 & 0.97 & 0.97 & 0.96 \\
             & (0.07) & (0.08) & (0.07) & (0.08) & (0.08) \\
    \% Black & 0.02 & 0.02 & 0.02 & 0.02 & 0.02 \\
             & (0.07) & (0.07) & (0.06) & (0.07) & (0.07) \\
    \% Married & 0.59 & 0.59 & 0.60 & 0.60 & 0.61 \\
               & (0.09) & (0.09) & (0.08) & (0.09) & (0.08) \\
    \% Separated & 0.01 & 0.01 & 0.01 & 0.01 & 0.01 \\
                 & (0.01) & (0.01) & (0.01) & (0.01) & (0.01) \\
    Income Heterogeneity Index & 0.10 & 0.10 & 0.09 & 0.09 & 0.09 \\
                               & (0.08) & (0.08) & (0.07) & (0.07) & (0.06) \\
    Median Family Income & 61,014 & 61,445 & 58,837 & 60,269 & 60,079 \\
                         & (17,694) & (18,283) & (14,177) & (15,687) & (14,143) \\
    \% Under 5 Years Old & 0.06 & 0.06 & 0.06 & 0.06 & 0.06 \\
                         & (0.02) & (0.02) & (0.02) & (0.02) & (0.02) \\
    \% Aged 5 to 17 & 0.20 & 0.20 & 0.21 & 0.20 & 0.20 \\
                    & (0.05) & (0.05) & (0.04) & (0.05) & (0.04) \\
    \% Aged 18 to 64 & 0.60 & 0.60 & 0.60 & 0.60 & 0.60 \\
                     & (0.05) & (0.05) & (0.05) & (0.05) & (0.04) \\
    \% Racial Minority & 0.04 & 0.04 & 0.03 & 0.03 & 0.04 \\
                       & (0.07) & (0.08) & (0.07) & (0.08) & (0.08) \\
    Number of Observations & 3,188 & 2,661 & 527 & 694 & 273 \\
    \\
    \multicolumn{6}{l}{\textbf{Panel B. Outcome Variables}} \\
    \textit{a) Median House Price} & & & & & \\
    t - 3 & 106,067 & 107,502 & 102,766 & 106,771 & 104,201 \\
          & (52,988) & (60,197) & (50,794) & (53,461) & (51,794) \\
    t - 2 & 105,528 & 107,211 & 101,456 & 106,207 & 103,713 \\
          & (53,242) & (60,893) & (48,438) & (54,162) & (50,786) \\
    t - 1 & 107,706 & 108,234 & 104,575 & 108,401 & 105,843 \\
          & (53,544) & (62,437) & (51,343) & (54,665) & (50,491) \\
    \textit{b) Median House Price (\%)} & & & & & \\
    t - 3 & 1,504 & 1,752 & 1,728 & 1,446 & 1,695 \\
          & (3,848) & (4,006) & (3,642) & (3,887) & (3,739) \\
    t - 2 & 1,429 & 1,734 & 1,680 & 1,388 & 1,562 \\
          & (3,693) & (3,978) & (3,612) & (3,769) & (3,462) \\
    t - 1 & 1,387 & 1,724 & 1,668 & 1,341 & 1,538 \\
          & (3,628) & (3,993) & (3,598) & (3,683) & (3,460) \\
    \textit{c) Median House Price per square foot} & & & & & \\
    t - 3 & 31,764 & 31,892 & 32,531 & 32,367 & 32,401 \\
          & (8,868) & (8,428) & (10,090) & (9,657) & (8,898) \\
    t - 2 & 31,714 & 32,083 & 32,048 & 31,571 & 32,166 \\
          & (8,630) & (8,392) & (9,224) & (8,723) & (8,372) \\
    t - 1 & 31,764 & 32,399 & 32,399 & 31,579 & 32,367 \\
          & (8,868) & (9,716) & (9,716) & (8,623) & (9,657) \\
\end{longtable}

% \begin{tablenotes}
%     \small
%     \item Notes: Panel A shows covariates used in estimating the treatment effect of failing to renew a road tax levy. Covariate means at the time of the vote are shown along with standard deviations in parentheses. The unit of observation is city-year. 'Income' is median family income in constant 2010 U.S. dollars. 'Global' refers to the full sample of city-year votes; 'Effective' refers to city-year votes within the mean estimated optimal bandwidth of 10 percentage points around the cutoff for years 1995 through 2021. The treatment group reflects cutting road funding by failing to renew the road tax levy. The control group reflects maintaining road funding by renewing the road tax levy. Panel B shows means of outcome variables for placebo years t - 3, t - 2, and t - 1.
% \end{tablenotes}


Table \ref{tab:variable_means_sd} shows that the treatment and control groups are well-balanced with respect to the covariates considered, suggesting that the groups are comparable and any observed differences in outcomes can more confidently be attributed to the treatment effect rather than to pre-existing differences.

