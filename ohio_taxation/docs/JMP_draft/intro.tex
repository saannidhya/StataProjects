Roads are an important form of infrastructure investment that affect firms’ production functions and people’s commuting costs. Many studies examine the importance of roads and how they affect residents' daily lives (\Citeauthor{currier2023} \citeyear{currier2023}; \Citeauthor{adukia2020} \citeyear{adukia2020}, \Citeauthor{banerjee2020} \citeyear{banerjee2020}; \Citeauthor{banerjee2020} \citeyear{banerjee2020}, \Citeauthor{banerjee2020} \citeyear{banerjee2020}). Some papers emphasize that building new roads increases employment and entry of new firms \citep{gibbons2019new}, thereby increasing economic activity. Other papers point out the potential for worse outcomes in the form of increased inequality between the rich and the poor \citep{hettige2006} and an exodus of workers seeking access to larger labor markets \citep{asher2020}.  
 
In this paper, we highlight the importance of maintaining road spending for cities in Ohio. We collect data on more than 3,000 road-tax levy renewal elections conducted within different county subdivisions (referred to as “cities”)  in Ohio. Using regression discontinuity design, we examine how cuts in maintenance spending influence key economic outcomes, focusing on cities where voting results were narrowly decided. Our results reveal that cutting maintenance spending by 34\% decreases house prices by \$20,729, with the effect starting four years after the vote and persisting through nine years after the vote. The \$20,729 discount represents about 12 percent of house value over the decade.  The delayed effect is consistent with the time it might take for roads to deteriorate noticeably.  We further find that these price effects are more pronounced in urban areas, where house prices fall by approximately \$25,825 on average, compared to no significant changes in rural areas. We also observe that higher-priced homes experience larger reductions in value relative to lower-priced homes, indicating a heterogeneous impact across different quantiles of home prices. On the other hand, we find little evidence for any overall change in wages or employment as a result of the decreased road maintenance spending for our full sample.  We do find, however, an effect on employment and wages in high-poverty cities.  Cities above the 75th percentile of poverty that cut road maintenance taxes have 17.8\% lower wages than otherwise similar cities that barely renew road tax levies.  High-poverty cities that cut road tax levies also lose about 11\% more jobs than those that successfully renew maintenance spending.  We speculate that the amenity value of attractive roads, changes in business productivity, and compensating wage differentials are probably responsible for the results, rather than changes in traffic noise or crime rates.

Perhaps the study most related to ours is \cite{asher2020}, which studies the impact of new roads on villages in India. Like our study, its identification strategy is regression discontinuity, although ours is sharp rather than the fuzzy form.  It argues the main obstacle to identification in prior studies is that the placement of new roads is usually correlated with economic (or political) characteristics rather than exogenous.  Its findings suggest this is a serious problem with the literature because, unlike prior studies, it finds no strong link between economic growth and new road placement, suggesting that the estimates of previous studies that find a link are driven by road placement in villages that are already growing. \cite{asher2020} touts its use of village-level rather than regional-level data.  We, too, look at economic outcomes at the level of village, city, and township, the most local levels of government.  A surprising finding of \cite{asher2020} is that investment in transportation infrastructure does not affect village incomes, assets, or agricultural output.  Its measure of assets is a village-level average of a series of binary indicators of ownership of a variety of assets, along with separate regressions for the presence of a ‘solid house’, refrigerator, and phone; whereas we study the effect of transportation spending on wages, employment, and housing sale prices.  Of course, our use of a developed geography contrasts with rural villages in India.  Our efforts to achieve identification focus on the maintenance of existing roads, which avoids the endogeneity of the placement of new roads.
