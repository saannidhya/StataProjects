Roads are an important form of infrastructure investment that affect firms’ production functions and people’s commuting costs. Many studies examine the importance of roads and how they affect residents' daily lives (\Citeauthor{currier2023} \citeyear{currier2023}; \Citeauthor{adukia2020} \citeyear{adukia2020}; \Citeauthor{asher2020} \citeyear{asher2020}). Some papers emphasize that building new roads increases employment and entry of new firms \citep{gibbons2019new}, thereby increasing economic activity. Other papers point out the potential for worse outcomes in the form of increased inequality between the rich and the poor \citep{hettige2006} and an exodus of workers seeking access to larger labor markets \citep{asher2020}.  
 
In this paper, we highlight the importance of maintaining road spending for cities in Ohio. We collect data on more than 3,000 renewal road-tax levy elections conducted within different county subdivisions, including townships, cities and villages, in Ohio (referred to as "cities" for brevity). Using dynamic regression discontinuity design, we examine how these exogeneous cuts in maintenance spending influence housing prices, focusing on cities where voting results were narrowly decided. Our results reveal that when a city cuts its renewal road taxes, it experiences a loss in maintenance funds, its road quality declines by 17\% and house prices decreases by around \$15,350, with the effect starting four years after the vote and persisting through many years after the vote. This discount in house prices represents about 9 percent of house value over the decade. The delayed effect is consistent with the time it takes for roads to deteriorate noticeably. We find that these price effects are driven by urban areas rather than rural areas. We also observe that higher-priced houses experience larger percentage reductions in value than lower-priced houses, indicating a heterogeneous impact across different quantiles. This suggests that higher-income households are more sensitive to road quality than lower-income households. 

% We further find that these price effects are delayed but clear in urban areas, where house prices fall by 7.6\% on average, compared to inconsistent changes in values of houses in rural areas. We also observe that higher-priced homes experience larger reductions in value relative to lower-priced homes, indicating a heterogeneous impact across different quantiles of home prices. This suggests that wealthier households are more sensitive to road quality, as a poor road in front of a high-priced home is more likely to be noticed than a poor road in front of a low-priced home.

