In this section, we describe our empirical strategy to estimate the causal effect of cutting road maintenance spending on different outcome variables. One key feature of our quasi-experiment design is the exogeneity of timing of the election. The timing of the election is determined by the natural expiration of the road maintenance tax levy, which is typically 5 years, and is not impacted by factors such as the prevailing economic conditions or whether a road tax levy was passed or failed in earlier years.

\subsection{Regression Discontinuity in Panel Data setting}

Suppose that local government in area $i$ and year $t$ has to conduct a referendum to renew their road tax levy. The referendum is conducted to determine whether the road maintenance tax levy should be renewed or not. Let $v_{it}$ be the vote share against the renewal tax levy and  $v^*$ be the threshold determining the result of the referedum (levy fails to renew if $v_{it} > v^*$). Let $F_{it} = 1(v_{it} > v^*)$ be an indicator to represent if the renewal road tax levy fails, and $y_{it}$ be some outcome variable of interest. We can write,

$$
y_{it} = \alpha + \theta F_{it} + \epsilon_{it}
$$

where $\alpha$ is the intercept, $\theta$ is the parameter of interest representing the causal effect of renewing a road tax levy and $\epsilon_{it}$ is the error term representing all other determinants of the outcome. Then, around a narrow enough window around the threshold $v^*$, we can estimate the causal effect of renewing a road tax levy on the outcome variable $y_{it}$ by comparing the outcome variable for cities that narrowly passed the referendum to those that narrowly failed the referendum \cite{lee_lemieux2010}.

\subsection{Intent-to-Treat (ITT) Estimator}

We follow a model of Regression Discontinuity Design (RDD) similar to \cite{cellini2010value} and estimate the {\bf Intent-to-Treat} or ITT estimator. We prefer using ITT estimator instead of the alternative {\bf Treatment on the Treated} (TOT) estimator because the ITT estimator is more suited to our setting given the independence of the renewal elections. As described in \cite{cellini2010value}, when the elections are independent, the ITT estimator equals the TOT estimator.

We operationalize our ITT estimator using the following regression equation -

$$
Y_{i,t+\tau} = \alpha_\tau + \kappa_t + F_{it} \theta_{\tau}^{ITT} + P_g (v_{it}, \gamma_\tau) + Z_{it} \beta_\tau + \epsilon_{i,t + \tau}
$$

Above, we have a municipality $i$ that held an election in year $t$ and we study this municipality's outcome $\tau$ years later. $Y_{i,t+\tau}$ represent the outcome variable for city, village or township $i$ at year $t + \tau$. The outcome variables we study include house prices, wages, and employment. We define treatment as failure of a city, village or township to renew its road maintenance tax levy, which is represented by the indicator $F_{it}$ and $\theta_{\tau}^{ITT}$ is the causal effect of failing to renew road tax on the outcome. $P_g (v_{it}, \gamma_\tau)$ is a polynomial function of the running variable $v_{it}$, which is the percent of votes against the renewal tax levy. $\alpha_\tau$ and $\kappa_t$ represent timing and year-specific fixed effects. We estimate $\theta_{\tau}^{ITT}$ by comparing the outcome variable for cities that narrowly passed the referendum to those that narrowly failed the referendum.        


We use the bandwidth selection method described in \cite{calonico2019regression} to find the mean optimal bandwidth $h$ and then conduct a local polynomial regression after choosing a weighting scheme $k$. The bandwidth $h$ determines size of the neighborhood around the cutoff $v^*$, where the neighborhood is $(v^* - h, v^* + h)$. Only observations with values of the running variable within this neighborhood are used to compute the treatment effect estimate $\hat{\tau}$ and, for a small enough neighborhood, the continuity assumption central to the RDD estimator is satisfied. The weighting scheme $k$ determines the weights of the observations within the neighborhood $(v^* - h, v^* + h)$ and is another important factor in determining $\theta_{\tau}$. The most common types of weighting schemes are uniform, triangular and Epanechnikov. We use the default Mean Squared Error Regression Discontinuity (MSERD) method to compute the effective bandwidth ($h$) and bias bandwidth ($b$) for each outcome variable. This method identifies the bandwidth that minimizes the trade-off between bias and variance of the treatment effect estimate. All observations are used to estimate $h$ and $b$, but only the observations within the selected bandwidth $h$ i.e. effective observations within $h$ percentage points on either side of the vote share cutoff $v^*$ are used to identify our treatment effect estimates $\theta_\tau$, for different $\tau$ years.

