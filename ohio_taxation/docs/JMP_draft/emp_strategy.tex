We follow a model of sharp Regression Discontinuity Design (RDD), as detailed in \cite{calonico2019regression}. A formal description of the model is provided below. Let $Y_{it}$ represent the outcome variable for city, village or township $i$ at year $t$. The outcome variables we study include house prices, wages, and employment. Drawing from the Rubin causal model, we use potential outcomes notation to describe $Y_{it}(1)$ as a potential outcome variable if exposed to treatment and $Y_{it}(0)$ as the potential outcome variable in absence of treatment i.e. our control. We define treatment as failure of a city, village or township to renew its road maintenance tax levy. Let $T_{it} \in \{0,1\}$ represent cities that receive treatment (1) or not (0) based on voting at year $t$, so that our realized outcome is represented by the following equation:

$$
Y_{it} = Y_{it}(1) \times T_{it} + Y_{it}(0) \times (1-T_{it})
$$

Let $X_{it}$ be the running variable that determines whether a realized outcome variable was treated or not based on some cutoff $\underbar{x}$. In our study, the running variable is the percent of votes against the renewal tax levy and the cutoff is 50 percent. Let $Z_{it}$ be our vector of covariates and $\tau$ be our parameter of interest i.e. the treatment effect. This parameter of interest measures the causal effect of cutting road maintenance spending on different outcome variables. Under the assumption of continuity at the cutoff $\underbar{x}$, the RDD allows us to compute the average treatment effect $\tau = \mathbb{E}(Y_{it}(1) - Y_{it}(0) | X_{it} = \underbar{x})$. We employ both local polynomial and local randomization approaches to compute the estimate for $\tau$, which we call $\hat{\tau}$.

To observe the persistence of treatment effects, we develop an event study-styled estimator that tracks the outcome variable for different values in future. The regression equation takes the form -

$$
Y_{i(t+r)} = \alpha_r + T_{i(t+r)} \tau_r + X_{i(t+r)} \beta_r + T_{i(t+r)} X_{i(t+r)} \theta_r + Z_{i(t+r)} \gamma_r + \epsilon_{i(t+r)}
$$

\noindent where $r \in \{-3,-2, ..., 10 \}$, $\alpha_r$ is the intercept, $\beta_r$ is the coefficient on the running variable $X_{i(t+r)}$, $\theta_r$ is the coefficient of the interaction term, $\gamma_r$ is a vector of coefficients for covariates matrix $Z_{i(t+r)}$ and $\epsilon_{i(t+r)}$ is the idiosyncratic error term. Our parameter of interest, $\tau_r$ measures the causal effect of cutting road maintenance spending on different outcome variables for different values of $r$. We estimate the treatment effect for different values of $r$ to observe the persistence of treatment effects and consider the possibly of placebo effects. 

The above equation has been described in detail in \cite{calonico2019regression} and is called the covariate-adjusted RDD estimator. We use the bandwidth selection method described in \cite{calonico2019regression} to choose a bandwidth $h$ and then conduct a local linear polynomial regression after choosing a weighting scheme. The bandwidth $h$ determines size of the neighborhood around the cutoff $\underbar{x}$, where the neighborhood is $(\underbar{x}-h, \underbar{x}+h)$. Only observations with values of the running variable within this neighborhood are used to compute the treatment effect estimate $\hat{\tau}$ and, for a small enough neighborhood, the continuity assumption central to the RDD estimator is satisfied. The weighting scheme $k$ determines the weights of the observations within the neighborhood $(\underbar{x}-h, \underbar{x}+h)$ and is another important factor in determining $\hat{\tau}$. The most common types of weighting schemes are uniform, triangular and Epanechnikov. We use the default Mean Squared Error Regression Discontinuity (MSERD) method to compute the effective bandwidth ($h$) and bias bandwidth ($b$) for each outcome variable. This method identifies the bandwidth that minimizes the trade-off between bias and variance of the treatment effect estimate. All observations are used to estimate $h$ and $b$, but only the observations within the selected bandwidth $h$ i.e. effective observations within $h$ percentage points on either side of the 0.50 vote share cutoff are used to identify our treatment effect estimates.

