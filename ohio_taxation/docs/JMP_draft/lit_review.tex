{\bf Related Literature.} Perhaps the study most related to ours is \cite{asher2020}, which studies the impact of new roads on villages in India. Like our study, its identification strategy is regression discontinuity, although ours is sharp rather than the fuzzy form.  It argues the main obstacle to identification in prior studies is that the placement of new roads is usually correlated with economic (or political) characteristics rather than exogenous. Its findings suggest this is a serious problem with the literature because, unlike prior studies, it finds no strong link between economic growth and new road placement, suggesting that the estimates of previous studies that find a link are driven by road placement in villages that are already growing. \cite{asher2020} touts its use of village-level rather than regional-level data.  We, too, look at economic outcomes at the level of village, city, and township, the most local levels of government.  A surprising finding of \cite{asher2020} is that investment in transportation infrastructure does not affect village incomes, assets, or agricultural output.  Its measure of assets is a village-level average of a series of binary indicators of ownership of a variety of assets, along with separate regressions for the presence of a ‘solid house’, refrigerator, and phone; whereas we study the effect of transportation spending on wages, employment, and housing sale prices.  Of course, our use of a developed geography contrasts with rural villages in India.  Our efforts to achieve identification focus on the maintenance of existing roads, which avoids the endogeneity of the placement of new roads.

Another study that is related to ours is \cite{cellini2010value}, which studies the effect of new capital projects for schools, funded via new local bond issues and raised by referendums, on house prices to deduce the value of school facility investments. Both our study and \cite{cellini2010value} employ a dynamic regression discontinuity design and analyze changes in regional property values. Moreover, both papers rely on broadly similar identification strategies and assignment mechanisms: in each case, votes in favor of or against a local referendum serves as the source of exogenous variation. Despite these similarities, a few key distinctions stand out. First, \cite{cellini2010value} draws on all observations in their sample, while we limit our analysis to the optimal bandwidth around the 50\% vote-share cutoff, even after using flexible controls. Second, \cite{cellini2010value} implements a fuzzy regression discontinuity approach, whereas ours is sharp, focusing exclusively on narrowly passing versus narrowly failing road-tax renewal votes. Lastly, we focus on the Intent-to-Treat (ITT) estimator, which focuses on the effect of referedum without controlling for results of subsequent referendums and it suits our setting given the independence of renewal referendums coming from exogeneous timing of such elections (more details in Section \ref{sec:empirical}). As \cite{cellini2010value} show, when elections are indeed independent, the ITT estimator is equal to the Treatment on the Treated (TOT) estimator. 

We highlight a substantial literature studying the effect of transportation infrastructure on house prices. \cite{hoogendoorn2019house} studies the effect of opening of a tunnel on house prices in the Netherlands, noting that prior research studying transportation infrastructure in developed geographies suffers from reverse causality.  It argues that the opening of the Westerscheldtunnel is a fairly exogenous event, with natural borders that prevent contamination of results by the surrounding environment.  It finds half the capitalized value of the tunnel occurs more than a year before the tunnel opens.  Our data, too, is for a developed nation. \cite{hoogendoorn2019house} argues that the exogeneity of the opening of the tunnel, along with hedonic controls, time trends and postcode fixed effects, identifies its estimates.  One novelty of our study is how ordinary the events are that we study.  While the opening of a new tunnel is important, it is rare.  Votes to renew infrastructure spending are common events in many local governments in the United States, and the quantity of road maintenance spending is regularly chosen by governments around the world, if not by voting then directly by bureaucrats.  It is therefore also important to study the effects of road maintenance spending on house prices. \cite{li2016wheels} studies the overall effect on apartment prices of new subway lines in Beijing, but the estimates may be a net effect of competing factors. \cite{gibbons2005valuing}, studying the construction of new rail stations for the London underground and light rail services, notes that the effect on house prices captures the net effect of better access, increased crime, and increased noise pollution. \cite{levkovich2016effects} looks at the effect of highway development on house prices in the Netherlands.  It separates out accessibility effects from noise pollution and increased traffic effects by looking at different neighborhoods near the highway development.  Its repeat sales difference in differences model finds increased house prices from anticipation effects \citep{kohlhase1991impact}. \cite{beenstock2016hedonic} also finds anticipation effects for house prices (but not new construction) for the development of a highway across Israel. 

{\bf Contribution.} Our paper contributes to the literature on three main fronts. First, we establish a new dataset for our quasi-experiment design, which allows us to study how local tax provision changes impact local infrastructure maintenance and neighborhood home prices. This dataset is unique in that, it contains referendums from local governments in Ohio, which are not typically collectively avaiable anywhere else and linked to local real estate information. This allows us to study the effects of these renewal tax cuts on home sale values at the local level. We also document a novel dataset of road images collected via Google Street Views API for Ohio, through which we are able to classify road quality.

Second, we focus on long-term effects of reduced road taxes funded mainly via property taxes, and observe statistically significant decline in house prices occurring after four years and persisting in later years. This delayed effect highlights how the consequences of road tax cuts are not immediate but gradually accumulate as funding is reduced and roads deteriorate. Our research also identifies heterogeneous impacts, showing that urban areas and higher-priced homes suffer more pronounced price declines. By focusing on distributional effects, the study adds nuance to the understanding of how infrastructure maintenance affects local housing markets.

Third, while much of the literature focuses on infrastructure development and expansion, where the impact of new infrastructure projects is one of the primary concerns and reverse causality is easier to address, we shift the focus to infrastructure maintenance in a developed economy and address endogeneity issues through our quasi-experiment setting. Our approach provides insights that are more relevant for policymakers in advanced economies, where road infrastructure is well established but requires continuous upkeep to avoid economic decline.

{\bf Roadmap.} The rest of the paper is organized as follows. Section \ref{sec:data} provides background information on the data and provides information about the variables used in the study. Section \ref{sec:empirical} outlines the empirical strategy. Section \ref{sec:result} presents the results of the study and shares the relevant robustness checks. Section \ref{sec:mech} suggests some mechanisms and Section \ref{sec:conclusion} concludes.