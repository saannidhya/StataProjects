We highlight a substantial literature studying the effect of transportation infrastructure on house prices. \cite{hoogendoorn2019house} studies the effect of opening of a tunnel on house prices in the Netherlands, noting that prior research studying transportation infrastructure in developed geographies suffers from reverse causality.  It argues that the opening of the Westerscheldtunnel is a fairly exogenous event, with natural borders that prevent contamination of results by the surrounding environment.  It finds half the capitalized value of the tunnel occurs more than a year before the tunnel opens.  Our data, too, is for a developed nation. \cite{hoogendoorn2019house} argues that the exogeneity of the opening of the tunnel, along with hedonic controls, time trends and postcode fixed effects, identifies its estimates.  One novelty of our study is how ordinary the events are that we study.  While the opening of a new tunnel is important, it is rare.  Votes to renew infrastructure spending are common events in many local governments in the United States, and the quantity of road maintenance spending is regularly chosen by governments around the world, if not by voting then directly by bureaucrats.  It is therefore also important to study the effects of road maintenance spending on house prices. \cite{li2016wheels} studies the overall effect on apartment prices of new subway lines in Beijing, but the estimates may be a net effect of competing factors. \cite{gibbons2005valuing}, studying the construction of new rail stations for the London underground and light rail services, notes that the effect on house prices captures the net effect of better access, increased crime, and increased noise pollution. \cite{levkovich2016effects} looks at the effect of highway development on house prices in the Netherlands.  It separates out accessibility effects from noise pollution and increased traffic effects by looking at different neighborhoods near the highway development.  Its repeat sales difference in differences model finds increased house prices from anticipation effects \citep{kohlhase1991impact}. \cite{beenstock2016hedonic} also finds anticipation effects for house prices (but not new construction) for the development of a highway across Israel. 

We also cite literature assessing the effect of transportation infrastructure on employment and wages. Gibbons et al 2019 analyze how new road infrastructure in Britain affects employment and labor productivity. It finds that while new roads increase area-level employment and the number of firms, they may lead to job reductions in existing firms, coupled with increased wages and output per worker \citep{duranton2012}. This study focuses on the long-term effects of transportation infrastructure on employment, highlighting how improved transit can shift the spatial concentration of industries and workers. Their findings show a significant relationship between transportation improvements and shifts in employment, especially in urban areas. Our results are also consistent with \cite{gibbons2019new} in that changes in road investment do not seem to be related to employment at the establishment level.  Neither \cite{dalenberg1995effects} nor \cite{gibbons2019new} look at the effect of transportation infrastructure in high-poverty areas specifically.  While we find no link between transportation infrastructure investment and wages or employment in our overall sample, we do find a link in high-poverty cities.

Our paper contributes to the literature on three main fronts. First, instead of studying the economic implications of constructing new roads, we look at the maintenance of existing roadways, a much more commonplace occurrence and therefore possibly more relevant to public policy. In fact, the exogenous shock we study, cutting maintenance spending, is essentially the opposite of the increased infrastructure more commonly studied.  
Second, while much of the literature focuses on developing nations, where infrastructure expansion is a primary concern and reverse causality is easier to address, we shift the focus to a developed economy, where the challenges are different and centered around the maintenance of aging infrastructure. This provides insights that are more relevant for policymakers in advanced economies, where road infrastructure is well established but requires continuous upkeep to avoid economic decline.
Third, we focus on long-term effects of reduced road maintenance on house prices, with statistically significant declines occurring only after four years and persisting through nine years. This delayed effect highlights how the consequences of infrastructure spending cuts are not immediate but gradually accumulate as roads deteriorate. Our research also identifies heterogeneous impacts, showing that urban areas and higher-priced homes suffer more pronounced price declines. By focusing on distributional effects, the study adds nuance to the understanding of how infrastructure maintenance affects local housing markets.

The rest of the paper is organized as follows. Section I provides background information on the data and provides information about the variables used in the study. Section II outlines the empirical strategy. Section III presents the results of the study and shares the relevant robustness checks. Section IV discusses mechanisms. Section V concludes.